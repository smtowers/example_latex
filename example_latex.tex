%%%%%%%%%%%%%%%%%%%%%%%%%%%%%%%%%%%%%%%%%%%%%%%%%%%%%%%%%%%%%%%%%%%%%%%%%%%%%%%%%%
%%%%%%%%%%%%%%%%%%%%%%%%%%%%%%%%%%%%%%%%%%%%%%%%%%%%%%%%%%%%%%%%%%%%%%%%%%%%%%%%%%
% example Latex document showing the layout of a paper, with examples of
% figures and tables, and other paper elements
%%%%%%%%%%%%%%%%%%%%%%%%%%%%%%%%%%%%%%%%%%%%%%%%%%%%%%%%%%%%%%%%%%%%%%%%%%%%%%%%%%
\documentclass[12pt]{article}
\usepackage{epsfig}
\usepackage{rotating}
\usepackage[table]{xcolor}
\definecolor{LightCyan}{rgb}{0.88,1,1}
\usepackage[none]{hyphenat}
%%%%%%%%%%%%%%%%%%%%%%%%%%%%%%%%%%%%%%%%%%%%%%%%%%%%%%%%%%%%%%%%%%%%%%%%%%%%%%%%%%
% allows multiple authors
%%%%%%%%%%%%%%%%%%%%%%%%%%%%%%%%%%%%%%%%%%%%%%%%%%%%%%%%%%%%%%%%%%%%%%%%%%%%%%%%%%
\usepackage{authblk}

%%%%%%%%%%%%%%%%%%%%%%%%%%%%%%%%%%%%%%%%%%%%%%%%%%%%%%%%%%%%%%%%%%%%%%%%%%%%%%%%%%
% stuff for making compartmental diagrams with the Latex tikz package
%%%%%%%%%%%%%%%%%%%%%%%%%%%%%%%%%%%%%%%%%%%%%%%%%%%%%%%%%%%%%%%%%%%%%%%%%%%%%%%%%%
\usepackage{bm}
\usepackage{fixltx2e}
\usepackage{tikz}
\renewcommand{\familydefault}{\sfdefault}
\DeclareMathSizes{10}{14}{11}{14}
\usetikzlibrary{shapes,arrows,positioning,calc}

%%%%%%%%%%%%%%%%%%%%%%%%%%%%%%%%%%%%%%%%%%%%%%%%%%%%%%%%%%%%%%%%%%%%%%%%%%%%%%%%%%
%%%%%%%%%%%%%%%%%%%%%%%%%%%%%%%%%%%%%%%%%%%%%%%%%%%%%%%%%%%%%%%%%%%%%%%%%%%%%%%%%%
% here are examples of how to define your own commands that you can
% use in Latex.  Within your analysis code, you can print out results
% formatted in \newcommand{} statements, such that you don't have to 
% transcribe data by hand into Latex... you can just copy and paste from your
% program output
%
% I always put a notation in my Latex file stating which R script produced the
% \newcommand statements, or figures, or tables.
% These lines were printed out by example_latex_output.R
%%%%%%%%%%%%%%%%%%%%%%%%%%%%%%%%%%%%%%%%%%%%%%%%%%%%%%%%%%%%%%%%%%%%%%%%%%%%%%%%%%
\newcommand{\yearmin}{$1995$}
\newcommand{\yearmax}{$2018$}
\newcommand{\StdDeviationOfData}{$0.998$} 

%%%%%%%%%%%%%%%%%%%%%%%%%%%%%%%%%%%%%%%%%%%%%%%%%%%%%%%%%%%%%%%%%%%%%%%%%%%%%%%%%%
%%%%%%%%%%%%%%%%%%%%%%%%%%%%%%%%%%%%%%%%%%%%%%%%%%%%%%%%%%%%%%%%%%%%%%%%%%%%%%%%%%
%%%%%%%%%%%%%%%%%%%%%%%%%%%%%%%%%%%%%%%%%%%%%%%%%%%%%%%%%%%%%%%%%%%%%%%%%%%%%%%%%%
\title{An example Latex document}

%\author{Sherry Towers \\
        %Simon A. Levin Mathematical, Computational and Modeling Sciences Center \\
        %Arizona State University\\
        %Tempe, AZ, U.S.A.
\author[1]{Author A\thanks{A.A@university.edu}}
\author[1]{Author B\thanks{B.B@university.edu}}
\author[1]{Author C\thanks{C.C@university.edu}}
\author[2]{Author D\thanks{D.D@university.edu}}
\author[2]{Author E\thanks{E.E@university.edu}}
\affil[1]{Department of Computer Science, \LaTeX\ University}
\affil[2]{Department of Mechanical Engineering, \LaTeX\ University}

\date{\today}

%%%%%%%%%%%%%%%%%%%%%%%%%%%%%%%%%%%%%%%%%%%%%%%%%%%%%%%%%%%%%%%%%%%%%%%%%%%%%%%%%%
%%%%%%%%%%%%%%%%%%%%%%%%%%%%%%%%%%%%%%%%%%%%%%%%%%%%%%%%%%%%%%%%%%%%%%%%%%%%%%%%%%
%%%%%%%%%%%%%%%%%%%%%%%%%%%%%%%%%%%%%%%%%%%%%%%%%%%%%%%%%%%%%%%%%%%%%%%%%%%%%%%%%%


\begin{document}
\maketitle

\begin{abstract}
Here is where you would put your Abstract.
This analysis examines data from \yearmin\ to \yearmax.

\end{abstract}
\pagebreak

%%%%%%%%%%%%%%%%%%%%%%%%%%%%%%%%%%%%%%%%%%%%%%%%%%%%%%%%%%%%%%%%%%%%%%%%%%%%%%%%%%
%%%%%%%%%%%%%%%%%%%%%%%%%%%%%%%%%%%%%%%%%%%%%%%%%%%%%%%%%%%%%%%%%%%%%%%%%%%%%%%%%%
%%%%%%%%%%%%%%%%%%%%%%%%%%%%%%%%%%%%%%%%%%%%%%%%%%%%%%%%%%%%%%%%%%%%%%%%%%%%%%%%%%
\section{Introduction}

Here is where you would put your Introduction\footnote{This is an example of a footnote}.

Here is an example of a citation to the Lacum {\it et al} (2014) paper
that describes the seven key elements
of scientific paper~\cite{lacum2014teaching}.
Other papers on this theme are References~\cite{van2016scientific,seiradakis2018training}.

Here is an example of how to insert an extra positive (or negative)
vertical space in the text:
\vspace*{1cm}

This space is separated 1cm below the preceding line.

Here is an example of how to insert an extra positive (or negative) horizontal \hspace*{1cm} space in the text.


%%%%%%%%%%%%%%%%%%%%%%%%%%%%%%%%%%%%%%%%%%%%%%%%%%%%%%%%%%%%%%%%%%%%%%%%%%%%%%%%%%
%%%%%%%%%%%%%%%%%%%%%%%%%%%%%%%%%%%%%%%%%%%%%%%%%%%%%%%%%%%%%%%%%%%%%%%%%%%%%%%%%%
%%%%%%%%%%%%%%%%%%%%%%%%%%%%%%%%%%%%%%%%%%%%%%%%%%%%%%%%%%%%%%%%%%%%%%%%%%%%%%%%%%
\section{Methods and Materials}
\subsection{Data}

If your analysis involves data, here is where you would thoroughly describe it.

This is an example of a bulletted itemized list:
\begin{itemize}
\item Item 1
\item Item 2
\item Item 3
\end{itemize}

Here is an example of an enumerated itemized list:
\begin{enumerate}
\item Item 1
\item Item 2
\item Item 3
\end{enumerate}

Here is an example of an descriptive itemized list:
\begin{enumerate}
\item[Item 1:] describe item 1
\item[Item 2:] describe item 2
\item[Item 3:] describe item 3
\end{enumerate}


%%%%%%%%%%%%%%%%%%%%%%%%%%%%%%%%%%%%%%%%%%%%%%%%%%%%%%%%%%%%%%%%%%%%%%%%%%%%%%%%%%
%%%%%%%%%%%%%%%%%%%%%%%%%%%%%%%%%%%%%%%%%%%%%%%%%%%%%%%%%%%%%%%%%%%%%%%%%%%%%%%%%%
%%%%%%%%%%%%%%%%%%%%%%%%%%%%%%%%%%%%%%%%%%%%%%%%%%%%%%%%%%%%%%%%%%%%%%%%%%%%%%%%%%
\subsection{Mathematical Model}
\label{sec:math_model}

If your analysis involves a mathematical model, here is where you would describe it.

This is an example of a set of equations:
\begin{eqnarray}
S^\prime & = & -\beta S I/N \nonumber \\
I^\prime & = & +\beta S I/N -\gamma I\nonumber \\
R^\prime & = & +\gamma I
\label{eqn:mymodel}
\end{eqnarray}
Here is an example of an inline equation $S^\prime = -\beta S I/N$.

Here is a reference to Equation~\ref{eqn:mymodel} in the text.  Note that when you
reference specific equations, tables, or paper sections in the text, you always 
capitalise those nouns.  Here is a reference to Section~\ref{sec:math_model}.


Here is an example of a compartmental model diagram, using the tikz package in 
Latex:
\tikzstyle{decision} = [diamond, draw, fill=blue!20,
    text width=4.5em, text badly centered, node distance=3cm, inner sep=0pt]
\tikzstyle{block} = [rectangle, draw, fill=blue!20, line width=0.5mm,
    text width=5em, text centered, rounded corners, minimum height=4em]
\tikzstyle{line} = [line width=0.5mm,draw, -latex']
\tikzstyle{cloud} = [draw, ellipse,fill=red!20, node distance=3cm,
    minimum height=2em]
\tikzstyle{myarrows}=[line width=1mm,draw=blue,-triangle 45,postaction={draw, line width=3mm, shorten >=4mm, -}]

\begin{tikzpicture}[node distance = 2cm, auto]
    % Place nodes
    \node [block] (S) {\bf S};
    \node [block, below=1.5cm of S] (E) {\bf E};
    \node [block, below=1.5cm of E] (I) {\bf I};
    \node [block, right=2.5cm of S] (Qs) {\bf Q\textsubscript{S}};
    \node [block, right=2.5cm of E] (Qe) {\bf Q\textsubscript{E}};
    \node [block, right=2.5cm of I] (H) {\bf H};
    \node [block, below right=1.5cm and 0.25cm of I] (R) {\bf R};
    \node [block, below=1.5cm of R] (D) {\bf D};
    \node [block, below=1.5cm of D] (B) {\bf B};


    %% Draw edges

    \path [line] ([yshift=-0.3cm]S.north east) -- node[above=5pt] {$\bm{(1-\omega)\alpha \rho {\sf I/N_2}}$} ([yshift=-0.3cm]Qs.north west);
    %\path [line] ([yshift=-0.3cm]S.north east) -- node[above=5pt] {$\bm{{{(1-\omega)\alpha \rho {\sf I}}\over{{\sf N_2}}}}$} ([yshift=-0.3cm]Qs.north west);

    \path [line] ([yshift=+0.3cm]Qs.south west) -- node {$\bm{\nu+\chi}$} ([yshift=+0.3cm]S.south east);

    \path [line] ([yshift=-0.3cm]E.north east) -- node[above=5pt] {$\bm{\omega\alpha \rho {\sf I/N}_2}$} ([yshift=-0.3cm]Qe.north west);
    \path [line] ([yshift=+0.3cm]Qe.south west) -- node {$\bm{\nu}$} ([yshift=+0.3cm]E.south east);

    \path [line] (S) -- node[left=1pt] {$\bm{(\lambda {\sf I} + \tau \lambda {\sf D})/{\sf N}_1}$} (E);

    \path [line] (E) -- node[left=1pt] {$\bm{\kappa}$} (I);
    \path [line] (Qe) -- node[right=1pt] {$\bm{\kappa}$} (H);

    \path [line] (I) -- node {$\bm{\rho}$} (H);

    \path [line] (I) -- node[left=1pt] {$\bm{\gamma}$} (R);

    \path [line] (H) -- node[right=1pt] {$\bm{\gamma}$} (R);

    \path [line] (I) -- node[left=1pt] {$\bm{\mu}$} (D.north west);

    \path [line] (H) -- node[right=1pt] {$\bm{\mu}$} (D.north east);

    \path [line] (D) -- node {$\bm{\delta}$} (B);

\end{tikzpicture}




%%%%%%%%%%%%%%%%%%%%%%%%%%%%%%%%%%%%%%%%%%%%%%%%%%%%%%%%%%%%%%%%%%%%%%%%%%%%%%%%%%
%%%%%%%%%%%%%%%%%%%%%%%%%%%%%%%%%%%%%%%%%%%%%%%%%%%%%%%%%%%%%%%%%%%%%%%%%%%%%%%%%%
%%%%%%%%%%%%%%%%%%%%%%%%%%%%%%%%%%%%%%%%%%%%%%%%%%%%%%%%%%%%%%%%%%%%%%%%%%%%%%%%%%
\subsection{Statistical methods}

If your analysis involves any kind of specialised statistical methodology, here is
where you would describe it.

To account for potential over-dispersion in count data involved our analyses, we utilized
Negative Binomial likelihood fits~\cite{lloyd2007maximum}. 
The probability mass function (PMF) of the Negative Binomial distribution for observing $k$
counts when $\lambda$ are expected is~\cite{lloyd2007maximum}
\begin{eqnarray}
f(k|\lambda,\alpha) & = &
{{\Gamma(\alpha+k)}\over{k! \Gamma(\alpha)}}
\left({{\lambda}\over{\lambda+\alpha}}\right)^k
\left(1+{{\lambda}\over{\alpha}}\right)^{-\alpha}
\hspace*{1cm} \lambda>0,\alpha>0,
\label{eqn:nb}
\end{eqnarray}
where $\alpha$ is the over-dispersion parameter.
The mean of the PMF is $\lambda$.
When $\alpha\rightarrow \infty$ the Poisson distribution is obtained, and when
$\alpha\rightarrow 0$ (i.e. highly over-dispersed data) 
the log-series distribution is obtained~\cite{lloyd2007maximum}.
The Akaike Information Criterion (AIC) was used for selection of the appropriate 
model~\cite{akaike1974new}.

Given a set of $M$ observations of the number of
violent crimes per day, $k_i$, with $i=1,...,M$
the likelihood of the observations is
\begin{eqnarray}
{{\cal{L}}} & = & \prod_{i=1}^{M} f(k_i|\lambda(t_i),\alpha),
\label{eqn:like}
\end{eqnarray}
where $\lambda(t_i)$ is the expected number of counts on day $t_i$.
The best-fit model values
used in the calculation of the $\lambda(t_i)$ are the values that maximize this
likelihood~\cite{cowan1998statistical}.


%%%%%%%%%%%%%%%%%%%%%%%%%%%%%%%%%%%%%%%%%%%%%%%%%%%%%%%%%%%%%%%%%%%%%%%%%%%%%%%%%%
%%%%%%%%%%%%%%%%%%%%%%%%%%%%%%%%%%%%%%%%%%%%%%%%%%%%%%%%%%%%%%%%%%%%%%%%%%%%%%%%%%
%%%%%%%%%%%%%%%%%%%%%%%%%%%%%%%%%%%%%%%%%%%%%%%%%%%%%%%%%%%%%%%%%%%%%%%%%%%%%%%%%%
\section{Results}

Here, without discussion, is where you present your results, either within a
paragraph, or as a figure, or a table.

Here is an example of a reference to Figure~\ref{fig:figure1} in the text.  Here
is an example to a reference to Table~\ref{tab:table1}, which is delimited by
vertical lines.  Table~\ref{tab:table2} is a rotated table in landscape format,
with every second line coloured blue to aid in readability.

The one standard deviation uncertainty in my data was~\StdDeviationOfData.

%%%%%%%%%%%%%%%%%%%%%%%%%%%%%%%%%%%%%%%%%%%%%%
%%%%%%%%%%%%%%%%%%%%%%%%%%%%%%%%%%%%%%%%%%%%%%
% Produced by example_latex.R
% It is always a good idea to put a notation
% stating which R script produced the figure
% or table results
%%%%%%%%%%%%%%%%%%%%%%%%%%%%%%%%%%%%%%%%%%%%%%
 \begin{figure}[h]
   \begin{center}
    \mbox{\put(-190,0){ \epsfxsize=13cm
           \epsffile{example_latex_histogram_plot.eps}
     }}
     \vspace*{-0.0cm}
  \caption{
      \label{fig:figure1}
Histogram of Normally distributed data
   }
\end{center}
\end{figure}
%%%%%%%%%%%%%%%%%%%%%%%%%%%%%%%%%%%%%%%%%%%%%%
%%%%%%%%%%%%%%%%%%%%%%%%%%%%%%%%%%%%%%%%%%%%%%
%%%%%%%%%%%%%%%%%%%%%%%%%%%%%%%%%%%%%%%%%%%%%%



%%%%%%%%%%%%%%%%%%%%%%%%%%%%%%%%%%%%%%%%%%%%%%%%%%%%%%%%%%%%%%%%%%%%%%%%%%%%%%%%%%
%%%%%%%%%%%%%%%%%%%%%%%%%%%%%%%%%%%%%%%%%%%%%%%%%%%%%%%%%%%%%%%%%%%%%%%%%%%%%%%%%%
%%%%%%%%%%%%%%%%%%%%%%%%%%%%%%%%%%%%%%%%%%%%%%%%%%%%%%%%%%%%%%%%%%%%%%%%%%%%%%%%%%
\begin{table}
\begin{center}
\caption{
Here is an example of a table delimited by vertical lines.
\hskip 2in}\label{tab:table1}
\begin{tabular}{|l|l|l|}
\hline
 & Percentage increase in casualties & p-value \\
\hline
Effect A   &    $17\%$ $[16\%,18\%]$  &($p\!<\!0.001$)   \\
Effect B   &    $74\%$ $[53\%,97\%]$ & ($p\!<\!0.001$)   \\
Effect C   &  $141\%$ $[93\%,197\%]$ & ($p\!<\!0.001$)   \\
\hline
\end{tabular}
\end{center}
\end{table}


%%%%%%%%%%%%%%%%%%%%%%%%%%%%%%%%%%%%%%%%%%%%%%%%%%%%%%%%%%%%%%%%%%%%%%%%%%%%%%%%%%
%%%%%%%%%%%%%%%%%%%%%%%%%%%%%%%%%%%%%%%%%%%%%%%%%%%%%%%%%%%%%%%%%%%%%%%%%%%%%%%%%%
%%%%%%%%%%%%%%%%%%%%%%%%%%%%%%%%%%%%%%%%%%%%%%%%%%%%%%%%%%%%%%%%%%%%%%%%%%%%%%%%%%
\begin{sidewaystable}
\begin{center}
\caption{
This is an example of a sideways table, with every second row coloured blue.
}\hspace*{1in}\label{tab:table2}
\begin{tabular}{lll}
\hline
 & Percentage increase in casualties &  p-value \\
\hline
\rowcolor{LightCyan}
 Each extra firearm used by perpetrator   &     $12\%$ $[7\%,17\%]$ & ($p\!<\!0.001$)   \\
    Use of assault rifle by perpetrator   &   $83\%$ $[59\%,109\%]$ & ($p\!<\!0.001$)   \\
\rowcolor{LightCyan}
             Perpetrator mental illness   &  $129\%$ $[83\%,184\%]$ & ($p\!<\!0.001$)   \\
                            End of FAWB   &  $176\%$ $[57\%,352\%]$ & ($p\!<\!0.001$)   \\
\rowcolor{LightCyan}
  Beginning of changed response tactics   & $-49\%$ $[-71\%,-16\%]$ &       ($p=0.010$)   \\
\hline
\end{tabular}
\end{center}
\end{sidewaystable}


%%%%%%%%%%%%%%%%%%%%%%%%%%%%%%%%%%%%%%%%%%%%%%%%%%%%%%%%%%%%%%%%%%%%%%%%%%%%%%%%%%
%%%%%%%%%%%%%%%%%%%%%%%%%%%%%%%%%%%%%%%%%%%%%%%%%%%%%%%%%%%%%%%%%%%%%%%%%%%%%%%%%%
% use clearpage if you want all plots and figures before this point to be
% output before the text that comes after the clearpage
%%%%%%%%%%%%%%%%%%%%%%%%%%%%%%%%%%%%%%%%%%%%%%%%%%%%%%%%%%%%%%%%%%%%%%%%%%%%%%%%%%
\clearpage


%%%%%%%%%%%%%%%%%%%%%%%%%%%%%%%%%%%%%%%%%%%%%%%%%%%%%%%%%%%%%%%%%%%%%%%%%%%%%%%%%%
%%%%%%%%%%%%%%%%%%%%%%%%%%%%%%%%%%%%%%%%%%%%%%%%%%%%%%%%%%%%%%%%%%%%%%%%%%%%%%%%%%
%%%%%%%%%%%%%%%%%%%%%%%%%%%%%%%%%%%%%%%%%%%%%%%%%%%%%%%%%%%%%%%%%%%%%%%%%%%%%%%%%%
\section{Discussion}

Here is where you would discuss your results, putting them in context of past research.
You would also discuss the weaknesses of the study methodology.

%%%%%%%%%%%%%%%%%%%%%%%%%%%%%%%%%%%%%%%%%%%%%%%%%%%%%%%%%%%%%%%%%%%%%%%%%%%%%%%%%%
%%%%%%%%%%%%%%%%%%%%%%%%%%%%%%%%%%%%%%%%%%%%%%%%%%%%%%%%%%%%%%%%%%%%%%%%%%%%%%%%%%
%%%%%%%%%%%%%%%%%%%%%%%%%%%%%%%%%%%%%%%%%%%%%%%%%%%%%%%%%%%%%%%%%%%%%%%%%%%%%%%%%%
\section{Summary}

Here is where you would briefly summarise the results of the study and the conclusions and implications that can be drawn.

%%%%%%%%%%%%%%%%%%%%%%%%%%%%%%%%%%%%%%%%%%%%%%%%%%%%%%%%%%%%%%%%%%%%%%%%%%%%%%%%%%
%%%%%%%%%%%%%%%%%%%%%%%%%%%%%%%%%%%%%%%%%%%%%%%%%%%%%%%%%%%%%%%%%%%%%%%%%%%%%%%%%%
% here is where you would put the name of your bibtex file (minus the .bib extension)
% the unsrt bibliography style sorts the citations by order they were referenced
% in the text
%%%%%%%%%%%%%%%%%%%%%%%%%%%%%%%%%%%%%%%%%%%%%%%%%%%%%%%%%%%%%%%%%%%%%%%%%%%%%%%%%%
\bibliographystyle{unsrt}
\bibliography{example_latex}

\end{document}
